\section{Related Work}
\label{sec:related}
%The following articles/web resources appear to be relevant: \cite{Berrabah2007}, \cite{Demuth2001}

Several researchers have proposed solutions for translating OCL to SQL. For example, in \cite{Berrabah2007}, the authors demonstrate an approach for generating event-condition-action (ECA) rules comprising SQL triggers and procedures from OCL constraints attached to a UML class diagram, when the latter is translated into a relational schema. In \cite{Demuth2001}, the authors propose using OCL-derived views in relational databases designed using UML, to check the integrity of the persisted data. In this work each OCL constraint is translated into a view in the relational database that contains reports of integrity violations. Such violations can be handled using different strategies including rolling back the offending transaction, triggering a data reconciliation action, or simply reporting the violation to application users. This approach has been implemented in the context of the OCL2SQL prototype\footnote{\url{http://dresden-ocl.sourceforge.net/usage/ocl22sql/}}. A similar approach is proposed by the authors of \cite{Marder2009}. In \cite{Heidenreich2007}, the authors propose a framework for translating OCL invariants into multiple query languages including SQL and XQuery using model-to-text transformations.

All the approaches above propose compile-time translation of OCL to SQL. By contrast, our approach proposes run-time generation and lazy evaluation of SQL statements. While compile-time translation is feasible for side-effect free OCL constraints that are evaluated against a homogeneous target (e.g. a relational database), for use-cases that involve querying and modifying models conforming to different technologies (e.g. a relational database and an EMF model), this approach is not applicable. Another novelty of the approach proposed in this paper is that it does not require a UML model that specifies the schema of the database, and as such, it can be used on existing databases that have not been developed in a UML-driven manner.